% verze šablony 1.1 (15.2.2013)
\documentclass[12pt,a4paper]{article}			% větší písmo, formát papíru, třída dokumentu, čísla stránek           

	% načtení balíčků	
	\usepackage[czech]{babel}
	\usepackage[utf8]{inputenc} 			% UTF-8 kódování zdrojáku
	\usepackage[T1]{fontenc}    			% kódování fontu pro ne-ASCII znaky
	\usepackage{amsmath}        			% lepší sázení matematiky
	\usepackage{amssymb}	    			% matematické symboly
	\usepackage[intoc]{nomencl}	    		% pomáhá vytvořit seznam použitých zkratek nebo termínů 
	\usepackage{graphicx}	    			% vkládání obrázků
	\usepackage{multicol}				% více-sloupcový layout
	\usepackage{array,multirow}			% row-span v tabulkách  
	\usepackage{marginnote}				% poznámky na straně 
	\usepackage{bigdelim}				% spolu s multirow umožňuje vertikální svorky u tabulek  
	\usepackage{tabularx}				% lepsi tabulky  
	\usepackage{datetime}				% práce s datumy 
	\usepackage[usenames,dvipsnames, table]{xcolor}	% barevné tabulky a zvyrazneni textu pres colorbox
	\usepackage{lastpage}				% umožňuje automaticky vložit celkový počet stránek v dokumentu 
	\usepackage[ps2pdf, unicode]{hyperref}  	% odkazy (musí být za ostatními balíčky)
	\usepackage{breakurl}				% zalamování URL (musí být pod hyperref)
  \usepackage{eurosym}        % potreba pro pouziti euro symbolu

	% nastavení okrajů
	\setlength\textwidth{145mm}     		% šířka řádků
	\setlength\textheight{247mm}    		% zmenšení spodního okraje stránky 
	\setlength\oddsidemargin{0mm} 			% levý okraj stránky
	\setlength\topmargin{0mm}			% horní okraj stránky
	\setlength\headsep{0mm}	        		% ------ || ---------
	\setlength\headheight{0mm}			% ------ || ---------

\begin{document}
	\begin{center}	
		\textsc{\Large Major Language Detection}
	\end{center}

  \section{Features}
  Our classifier takes advantage of a total of 143 features (not including additional 96 mixed features) divided into four groups
  as follows: \textit{language-specific features}, \textit{generic language features}, \textit{non-language
  features} and \textit{mixed features}. In this section we briefly characterize each of these
  groups.

  \subsection{Language-specific features}

  The classifier employs 110 language-specific features falling into 11 different trait groups. Within each group a
  particular trait of the site is examined for each of 10 language classes separately. Selection of
  inspected language traits follows: 
  \begin{itemize}
    \item \textit{on-page language detection} - E.g. feature named \texttt{on\_page\_CS} determines the
      probability of a web page being in the Czech language based on analysis of all the text
      contained in the page being scrutinized. Similarly \texttt{on\_page\_DE} determines the
      probability for German language etc. Besides the above-described all-text analysis, the on-page 
      detector is also used for analysing language of \textit{site-wide} text (\texttt{SWT}) and
    \textit{non-site-wide} text (\texttt{non-SWT}) parts of the document separately as site-wide texts 
      such as menus may tend to be in different language than site-specific texts.
    \item \textit{backlink language prediction} - probabilities of the page being in a particular
      language based on language characteristics of other pages referencing it (e.g.
      \texttt{backlink\_EN})
    \item \textit{language statistics aggregated for domains} - language statistics for the
      containing domains of different levels
  \item \textit{language specifications in} HTML - such as \texttt{<HTML lang=...>} or \texttt{<HTML
      meta http-equiv="Content-Language" ...>}
    \item \text{listed language indicative tokens in URL} such as \texttt{cs} in:

    \texttt{http://www.europarl.europa.eu/news/}\underline{\texttt{cs}} 
  \item HTTP \textit{header field \texttt{Content-language}}
  \end{itemize}

  \subsection{Generic language features}
    For feature generation purposes, one can examine each web page from 10 different perspecitves -
    each corresponding to one particular language (i.e. a class). With generic language features, we
    take advantage of this fact by shifting from questions such as \uv{What is the probability that 
    this particular page is in the Czech language based on backlink statistics?} to questions 
    such as \uv{What is the probability that this page is in the language we are currently checking 
    based on backlink statistics?} Technically, this means simply
    that for each \textit{view} (i.e. particular language), we copy values of language-specific
    features for that language and use them as generic language features for the current view. Thus
    we end up with 11 generic language features (one for each trait group of language-specific
    features) such as \texttt{on\_page\_lang}, \texttt{backlink\_lang}, etc.

  \subsection{Non-language features}
    Not all used features are directly related to document language. Features in this group may help the
    classifier to first split the data space into distinct categories exhibiting different data patterns 
    to subsequently enhance the language detection itself. 

    \begin{itemize}
      \item \textit{type of site} - probabilities of the site being an e-shop, photo gallery etc.
      \item \textit{prices and currencies} - total numbers of matches of a set of regular
      expressions designed to detect prices in particular currencies (e.g. \texttt{\euro 35}) 
      \item \textit{text length} 
      \item \textit{backlink count}
    \end{itemize}
    
  \subsection{Mixed features}

    To make the job easier for the classifier, we also used mixed features which combine previously
    described features in various ways. Each of mixed features is either a sum or a product of a
    pair or a triplet of some selected features.


\end{document}
