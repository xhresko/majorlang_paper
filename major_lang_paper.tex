% THIS IS SIGPROC-SP.TEX - VERSION 3.1
% WORKS WITH V3.2SP OF ACM_PROC_ARTICLE-SP.CLS
% APRIL 2009
%
% It is an example file showing how to use the 'acm_proc_article-sp.cls' V3.2SP
% LaTeX2e document class file for Conference Proceedings submissions.
% ----------------------------------------------------------------------------------------------------------------
% This .tex file (and associated .cls V3.2SP) *DOES NOT* produce:
%       1) The Permission Statement
%       2) The Conference (location) Info information
%       3) The Copyright Line with ACM data
%       4) Page numbering
% ---------------------------------------------------------------------------------------------------------------
% It is an example which *does* use the .bib file (from which the .bbl file
% is produced).
% REMEMBER HOWEVER: After having produced the .bbl file,
% and prior to final submission,
% you need to 'insert'  your .bbl file into your source .tex file so as to provide
% ONE 'self-contained' source file.
%
% Questions regarding SIGS should be sent to
% Adrienne Griscti ---> griscti@acm.org
%
% Questions/suggestions regarding the guidelines, .tex and .cls files, etc. to
% Gerald Murray ---> murray@hq.acm.org
%
% For tracking purposes - this is V3.1SP - APRIL 2009

\documentclass{acm_proc_article-sp}

\begin{document}

\title{Major Language Detection}
%
% You need the command \numberofauthors to handle the 'placement
% and alignment' of the authors beneath the title.
%
% For aesthetic reasons, we recommend 'three authors at a time'
% i.e. three 'name/affiliation blocks' be placed beneath the title.
%
% NOTE: You are NOT restricted in how many 'rows' of
% "name/affiliations" may appear. We just ask that you restrict
% the number of 'columns' to three.
%
% Because of the available 'opening page real-estate'
% we ask you to refrain from putting more than six authors
% (two rows with three columns) beneath the article title.
% More than six makes the first-page appear very cluttered indeed.
%
% Use the \alignauthor commands to handle the names
% and affiliations for an 'aesthetic maximum' of six authors.
% Add names, affiliations, addresses for
% the seventh etc. author(s) as the argument for the
% \additionalauthors command.
% These 'additional authors' will be output/set for you
% without further effort on your part as the last section in
% the body of your article BEFORE References or any Appendices.

\numberofauthors{8} %  in this sample file, there are a *total*
% of EIGHT authors. SIX appear on the 'first-page' (for formatting
% reasons) and the remaining two appear in the \additionalauthors section.
%
\author{
% You can go ahead and credit any number of authors here,
% e.g. one 'row of three' or two rows (consisting of one row of three
% and a second row of one, two or three).
%
% The command \alignauthor (no curly braces needed) should
% precede each author name, affiliation/snail-mail address and
% e-mail address. Additionally, tag each line of
% affiliation/address with \affaddr, and tag the
% e-mail address with \email.
%
\alignauthor
Juraj Hre\v{s}ko\titlenote{Chief officer of Jurajsk\'{y} park.},Vojt\v{e}ch Diatka\titlenote{The Language Master.}, Kry\v{s}tof Bo\v{r}kovec
       \affaddr{Seznam.cz, a.s.}\\
       \affaddr{Radlick\'{a} 3294/10}\\
       \affaddr{Prague 5, Czech Republic}\\
       \email{\{juraj.hresko,vojtech.diatka,krystof.borkovec\}\\@firma.seznam.cz}
}

\maketitle
\begin{abstract}
TODO
\end{abstract}
\keywords{language detection} % NOT required for Proceedings

\section{Introduction}
  Our main goal was to choose Major Language given on-page language detection results and other features. Finding out languages and their proportion on a webpage is well-established topic in computational processing of web pages (citace). However, determining which of them is major one is not.

  Major Language of a given webpage is the intended language of all potential queries aiming at a given webpage. 

  Intended language is the language of web pages in which a user intends to find results in SERP. This definition implies that if a user searches for a webpages in a language x, our major goal is to provide webpages in language x.
  It is essential topic for any full text search engine to return webpages relevant to a query. One of the dimensions of relevancy is also a language. For each webpage, we thus need to ask: “If the user wanted to find this page, in which language would she formulate the query?” Our interpretation of major language presupposes that user knows what should be the language of the answer to her query.
  This definition strives to exclude any ambiguity. There are very few contexts in which there would be more queries in various languages aiming at the same page.
  We approached this topic with a machine-learning method based on our implementation of multiple additive oblivious decision trees called RC-rank.


  Our paper has X sections:
  \begin{enumerate}
  \item Introduction
  \item Task (Popis úlohy – kolik jazyků a které (a proč tyto jazyky), na jakou množinu aplikujeme klasifikátor, jaký je poměr jazyků tam…..)

  \item Related work (problem se neřeší tak často, ale zakládáme se na onpage detekci)

  \item Classifier  (co je zač a jak jsme si vyhráli s fíčurama)
  \item Feature description
  \item Training set
  \item Results
  \item Comparison with other majorLang algorithms
  \end{enumerate}

  Our task resembles a multi-category classification problem. 
  However this approach brings a number of issues arising primarily from the fact of class multiplicity. 
  In the first place, there is a problem with sampling as we needed to keep corresponding proportions 
  of real distribution of languages across our dataset. 
  This leads to the situation when some classes examples are quite rare and so their count is insufficient 
  for training proper classifier. We handled this aspect of the task by joining all positive samples 
  into one class and modifying the features by creating a special group of so called \textit{generic language features}. 
  Using this approach we were able to create more general classifier, which works as a ranker of all considered languages. 
  The final model was trained using RC-Rank algorithm which is based on multiple additive oblivious decision trees [cit. MM]. 

  \section{Features}
  Our classifier takes advantage of a total of 143 features (not including additional 96 mixed features) divided into four groups
  as follows: \textit{language-specific features}, \textit{generic language features}, \textit{non-language
  features} and \textit{mixed features}. In this section we briefly characterize each of these
  groups.

\subsection{Language-specific features}

  The classifier employs 110 language-specific features falling into 11 different trait groups. Within each group a
  particular trait of a site is examined for each of 10 language classes separately. Selection of
  inspected language traits follows: 
  \begin{itemize}
    \item \textit{on-page language detection} - E.g. feature named \texttt{on\_page\_CS} determines the
      probability of a web page being in the Czech language based on analysis of all the text
      contained on the page being scrutinized. Similarly \texttt{on\_page\_DE} determines the
      probability for German language etc. Besides the above-described all-text analysis, the \textit{on-page 
      detector} is also used for analysing language of \textit{site-wide} text part (\texttt{SWT}) and
    \textit{non-site-wide} text (\texttt{non-SWT}) part of the document separately as site-wide texts 
      such as menus may tend to be in different language than site-specific texts such as articles' bodies.
    \item \textit{backlink language prediction} - probabilities of the page being in a particular
      language based on language characteristics of other pages referencing it (e.g.
      \texttt{backlink\_EN})
    \item \textit{language statistics aggregated for domains} - language statistics for the
      containing domains of different levels
    \item \textit{language specifications in} HTML - such as \texttt{<HTML lang=...>} or \texttt{<HTML
      meta http-equiv="Content-Language"  ...>}
    \item \text{selected language indicative tokens in URL} such as \texttt{cs} in:

    \texttt{http://www.europarl.europa.eu/news/}\underline{\texttt{cs}} 
    \item HTTP \textit{header field} \texttt{Content-language}
  \end{itemize}


\subsection{Generic language features}
    For feature generation purposes, one can examine each web page from 10 different perspecitves -
    each corresponding to one particular language (i.e. a class). With generic language features, we
    take advantage of this fact by shifting from asking questions such as \textit{what is the probability that 
    this particular page is in the Czech language based on backlink statistics?} to questions 
    such as \textit{what is the probability that this page is in the language we are currently checking 
    based on backlink statistics?} Technically, this means simply
    that for each \textit{view} (i.e. particular language), we copy values of language-specific
    features for that language and use them as generic language features for the current view. Thus
    we end up with 11 generic language features (one for each trait group of language-specific
    features) such as \texttt{on\_page\_lang}, \texttt{backlink\_lang}, etc.
 \subsection{Non-language features}
    Not all used features are directly related to document language. Features in this group may help the
    classifier to first split the data space into distinct categories exhibiting different data patterns 
    to subsequently enhance the language detection itself. 

    \begin{itemize}
      \item \textit{type of site} - probabilities of the site being an e-shop, photo gallery etc.
      \item \textit{prices and currencies} - total numbers of matches of a set of regular
      expressions designed to detect prices in particular currencies (e.g. \texttt{\$35}) 
      \item \textit{text length} 
      \item \textit{backlink count}
    \end{itemize}
    
  \subsection{Mixed features}

    To make the job easier for the classifier, we also used mixed features which combine previously
    described features in various ways. Each of mixed features is either a sum or a product of a
    pair or a triplet of some selected features.

\balancecolumns

% That's all folks!
\end{document}
